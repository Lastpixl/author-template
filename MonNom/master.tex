% Pensez à indiquer english ici pour gérer les césures correctement
% Merci de ne conserver cette ligne pour éviter que le paramètre ne soit écrasé par un article précédent dans les actes
\selectlanguage{french} % valeurs possibles : french, english

\settitle[Sécurité des trotinettes]{Sécurité des trotinettes: prise de contrôle des roues à distance}

\setauthor[M.~MonNom, A.~MonAutreNom]{MonPrenom MonNom \and AutrePrenom AutreNom\\
  \email{monprenom.monnom@mycorp.com\\autreprenom.autrenom@mycorp.com}}

\institute{MyCorp}

% Pour gérer des organismes d'appartenance multiples, voici le modèle:
%
%   \setauthor[M.~MonNom, A.~MonAutreNom]{MonPrenom MonNom\inst{1} \and AutrePrenom AutreNom\inst{2}\\
%     \email{monprenom.monnom@mycorp.com\\autreprenom.autrenom@mycorp.com}}
%
%   \institute{OrganismePourMonNom \and OrganismePourAutreNom}
%
% Dans \institute, la numérotation se fera automatiquement et commence à 1


\maketitle
\index{MonNom, M.}
\index{MonAutreNom, A.}

\begin{abstract}
  Lorem ipsum dolor sit amet, consectetur adipisicing elit, sed do
  eiusmod tempor incididunt ut labore et dolore magna aliqua. Ut enim ad
  minim veniam, quis nostrud exercitation ullamco laboris nisi ut
  aliquip ex ea commodo consequat. Duis aute irure dolor in
  reprehenderit in voluptate velit esse cillum dolore eu fugiat nulla
  pariatur. Excepteur sint occaecat cupidatat non proident, sunt in
  culpa qui officia deserunt mollit anim id est laborum.
\end{abstract}


\section{Introduction}

\subsection{État de l'art}

Lorem ipsum dolor sit amet, consectetur adipisicing elit, sed do
eiusmod tempor incididunt ut labore et dolore magna aliqua. Ut enim ad
minim veniam, quis nostrud exercitation ullamco laboris nisi ut
aliquip ex ea commodo consequat. Duis aute irure dolor in
reprehenderit in voluptate velit esse cillum dolore eu fugiat nulla
pariatur. Excepteur sint occaecat cupidatat non proident, sunt in
culpa qui officia deserunt mollit anim id est laborum. Lorem ipsum
dolor sit amet, consectetur adipisicing elit, sed do eiusmod tempor
incididunt ut labore et \emph{dolore} magna aliqua. Ut enim ad minim
veniam, quis nostrud exercitation ullamco laboris nisi ut aliquip ex
ea commodo consequat. Duis aute irure dolor in reprehenderit in
voluptate velit esse cillum dolore eu fugiat nulla pariatur. Excepteur
sint occaecat cupidatat non proident, sunt in culpa qui officia
deserunt mollit anim id est laborum.

% Exemple d'inclusion d'une figure
% (avec une version couleur)

\begin{figure}[ht]
  \centering
  \ifssticbw
    \includegraphics[width=0.4\textwidth]{MonNom/img/bw-archi}
  \else
    \includegraphics[width=0.4\textwidth]{MonNom/img/archi}
  \fi
  \caption{Légende de l'image}
  \label{fig:monnom:archi}
\end{figure}

Lorem ipsum dolor sit amet, consectetur adipisicing elit, sed do
eiusmod tempor incididunt ut labore et dolore magna aliqua. Ut enim ad
minim veniam, quis nostrud exercitation ullamco laboris nisi ut
aliquip ex ea commodo consequat. Duis aute irure dolor in
reprehenderit in voluptate velit esse cillum dolore eu fugiat nulla
pariatur. Excepteur sint occaecat cupidatat non proident, sunt in
culpa qui officia deserunt mollit anim id est laborum. Lorem ipsum
dolor sit amet, consectetur adipisicing elit, sed do eiusmod tempor
incididunt ut labore et dolore magna aliqua. Ut enim ad minim veniam,
quis nostrud exercitation ullamco laboris nisi ut aliquip ex ea
commodo consequat. Duis aute irure dolor in reprehenderit in voluptate
velit esse cillum dolore eu fugiat nulla pariatur. Excepteur sint
occaecat cupidatat non proident, sunt in culpa qui officia deserunt
mollit anim id est laborum.


% Exemples de citations

Je sais aussi faire des citations, dans l'article de Charlie
Lembrouille, \cite{monnom:charlielembrouille}, il est démontré que, je cite,
\og{}les trotinettes sont vulnérables aux attaques par XSS\fg{}. C'est
moche, l'image~\ref{fig:monnom:archi} en étant la preuve !

Si une citation est manquante (comme \cite{monnom:referencefoireuse}),
un warning en rouge s'affichera lors de la compilation.


% Exemples de listes  à points

Avec des listes partout :
\begin{itemize}
\item Lorem ipsum dolor sit amet
  \begin{itemize}
  \item Parce que je le vaux bien
  \item N'est-ce pas ?
  \end{itemize}
\item Consectetur adipisicing elit
\item Sed do eiusmod tempor
\end{itemize}

Lorem ipsum dolor sit amet, consectetur adipisicing elit, sed do
eiusmod tempor incididunt ut labore et dolore magna aliqua. Ut enim ad
minim veniam, quis nostrud exercitation ullamco laboris nisi ut
aliquip ex ea commodo consequat. Duis aute irure dolor in
reprehenderit in voluptate velit esse cillum dolore eu fugiat nulla
pariatur. Excepteur sint occaecat cupidatat non proident, sunt in
culpa qui officia deserunt mollit anim id est laborum. Lorem ipsum
dolor sit amet, consectetur adipisicing elit, sed do eiusmod tempor
incididunt ut labore et dolore magna aliqua. Ut enim ad minim veniam,
quis nostrud exercitation ullamco laboris nisi ut aliquip ex ea
commodo consequat. Duis aute irure dolor in reprehenderit in voluptate
velit esse cillum dolore eu fugiat nulla pariatur. Excepteur sint
occaecat cupidatat non proident, sunt in culpa qui officia deserunt
mollit anim id est laborum.

\section{Use the source, Luke}

% Exemple d'inclusion d'extraits de code

Des fois, je peux aussi inclure du texte brut, comme la sortie
standard en figure~\ref{lst:monnom:outhello} qui a été généré par le
code~\ref{lst:monnom:helloword}.

\begin{lstlisting}[language={},caption={Sortie standard},label={lst:monnom:outhello}]
Hello world everybody
\end{lstlisting}

\begin{lstlisting}[language={Python},caption={Mon premier code},label={lst:monnom:helloword}]
#! /usr/bin/env python

import sys

def hello(name):
    print 'Hello world %s' % (name)

if __name__ == '__main__':
    hello(sys.argv[1] if len(sys.argv) > 1 else 'everybody')
\end{lstlisting}

% Exemple d'inclusion d'un tableau

Il est également possible d'inclure des tableaux comme le
tableau~\ref{tbl:monnom}, dont la numérotation se fait à part.

\begin{table}[ht]
  \centering
  \begin{tabular}{|c|}
  \hline
  Ligne 1 \\
  Ligne 2 \\
  ... \\
  Ligne n \\
  \hline
  \end{tabular}
  \caption{Légende du tableau}
  \label{tbl:monnom}
\end{table}

% Exemple d'inclusion d'un algorithme

Enfin, il existe le \emph{package} \texttt{algorithm} permet d'inclure
des algorithmes (voir~\ref{alg:monnom} pour un exemple).

\begin{algorithm}
\begin{algorithmic}[1]
  \Statex \textbf{Input:} $x$
  \Statex \textbf{Output:} $x+1$
  \State Ajouter un!
\end{algorithmic}
\caption{Mon bel algo}\label{alg:monnom}
\end{algorithm}


\bibliography{MonNom/biblio}

% Quelques éléments complémentaires de typo
%
% En français, il faut un espace insécable avant les signes doubles
% (:;?!) que vous pouvez insérer avec ~ mais il est plus simple de
% laisser LaTeX faire en ne collant le signe double.
%
% Pensez à coller vos \footnote au texte qui appelle la note de bas de
% page.
%
% Il faut mettre un espace insécable ~ avant les références~\ref{...} ou
% les citations~\cite{...}.
