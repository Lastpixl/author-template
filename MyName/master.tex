% You must put english here to handle hyphenation correctly
% Please keep this line to avoid using the parameter from the previous article in the proceedings
\selectlanguage{english} % possible values : french, english

\settitle[Security of trotinettes]{Security of \emph{trotinettes}: remote control access}

\setauthor[M.~MyName, A.~MyOtherName]{MyFirstName MyName \and AnotherFirstName MyOtherName\\
  \email{myfirstname.myname@mycorp.com\\a.myothername@mycorp.com}}

\institute{MyCorp}

% To handle authors with multiple institutions, you can use the following code:
%
%   \setauthor[M.~MonNom, A.~MonAutreNom]{MonPrenom MonNom\inst{1} \and AutrePrenom AutreNom\inst{2}\\
%     \email{monprenom.monnom@mycorp.com\\autreprenom.autrenom@mycorp.com}}
%
%   \institute{OrganismePourMonNom \and OrganismePourAutreNom}
%
% For \institute, the numbering is automatic (and starts at 1)


\maketitle
\index{MyName, M.}
\index{MyOtherName, A.}

\begin{abstract}
  Lorem ipsum dolor sit amet, consectetur adipisicing elit, sed do
  eiusmod tempor incididunt ut labore et dolore magna aliqua. Ut enim ad
  minim veniam, quis nostrud exercitation ullamco laboris nisi ut
  aliquip ex ea commodo consequat. Duis aute irure dolor in
  reprehenderit in voluptate velit esse cillum dolore eu fugiat nulla
  pariatur. Excepteur sint occaecat cupidatat non proident, sunt in
  culpa qui officia deserunt mollit anim id est laborum.
\end{abstract}


\section{Introduction}

\subsection{State of the art}

Lorem ipsum dolor sit amet, consectetur adipisicing elit, sed do
eiusmod tempor incididunt ut labore et dolore magna aliqua. Ut enim ad
minim veniam, quis nostrud exercitation ullamco laboris nisi ut
aliquip ex ea commodo consequat. Duis aute irure dolor in
reprehenderit in voluptate velit esse cillum dolore eu fugiat nulla
pariatur. Excepteur sint occaecat cupidatat non proident, sunt in
culpa qui officia deserunt mollit anim id est laborum. Lorem ipsum
dolor sit amet, consectetur adipisicing elit, sed do eiusmod tempor
incididunt ut labore et \emph{dolore} magna aliqua. Ut enim ad minim
veniam, quis nostrud exercitation ullamco laboris nisi ut aliquip ex
ea commodo consequat. Duis aute irure dolor in reprehenderit in
voluptate velit esse cillum dolore eu fugiat nulla pariatur. Excepteur
sint occaecat cupidatat non proident, sunt in culpa qui officia
deserunt mollit anim id est laborum.

% Figure sample
% (with both a B&W and a color version)

\begin{figure}[ht]
  \centering
  \ifssticbw
    \includegraphics[width=0.4\textwidth]{MyName/img/bw-archi}
  \else
    \includegraphics[width=0.4\textwidth]{MyName/img/archi}
  \fi
  \caption{Figure caption}
  \label{fig:myname:archi}
\end{figure}

Lorem ipsum dolor sit amet, consectetur adipisicing elit, sed do
eiusmod tempor incididunt ut labore et dolore magna aliqua. Ut enim ad
minim veniam, quis nostrud exercitation ullamco laboris nisi ut
aliquip ex ea commodo consequat. Duis aute irure dolor in
reprehenderit in voluptate velit esse cillum dolore eu fugiat nulla
pariatur. Excepteur sint occaecat cupidatat non proident, sunt in
culpa qui officia deserunt mollit anim id est laborum. Lorem ipsum
dolor sit amet, consectetur adipisicing elit, sed do eiusmod tempor
incididunt ut labore et dolore magna aliqua. Ut enim ad minim veniam,
quis nostrud exercitation ullamco laboris nisi ut aliquip ex ea
commodo consequat. Duis aute irure dolor in reprehenderit in voluptate
velit esse cillum dolore eu fugiat nulla pariatur. Excepteur sint
occaecat cupidatat non proident, sunt in culpa qui officia deserunt
mollit anim id est laborum.


% Quoting examples

I can also quote external references, e.g.~ with Charlie Lembrouille's
famous article~\cite{myname:charlielembrouille}, he proved that "we
cannot rely on our \emph{trotinettes} since they are vulnerable to XSS
attacks". It's sad, and figure~\ref{fig:myname:archi} illustrates the
situation!

If a reference is missing (like~\cite{myname:referencefoireuse}), a
red warning will be displayed during the compilation.


% Itemize samples

You can also use lists:
\begin{itemize}
\item Lorem ipsum dolor sit amet
  \begin{itemize}
  \item Because I can
  \item One more time
  \end{itemize}
\item Consectetur adipisicing elit
\item Sed do eiusmod tempor
\end{itemize}

Lorem ipsum dolor sit amet, consectetur adipisicing elit, sed do
eiusmod tempor incididunt ut labore et dolore magna aliqua. Ut enim ad
minim veniam, quis nostrud exercitation ullamco laboris nisi ut
aliquip ex ea commodo consequat. Duis aute irure dolor in
reprehenderit in voluptate velit esse cillum dolore eu fugiat nulla
pariatur. Excepteur sint occaecat cupidatat non proident, sunt in
culpa qui officia deserunt mollit anim id est laborum. Lorem ipsum
dolor sit amet, consectetur adipisicing elit, sed do eiusmod tempor
incididunt ut labore et dolore magna aliqua. Ut enim ad minim veniam,
quis nostrud exercitation ullamco laboris nisi ut aliquip ex ea
commodo consequat. Duis aute irure dolor in reprehenderit in voluptate
velit esse cillum dolore eu fugiat nulla pariatur. Excepteur sint
occaecat cupidatat non proident, sunt in culpa qui officia deserunt
mollit anim id est laborum.

\section{Use the source, Luke}

% Raw test inclusion

Sometimes you might want to include raw text, like the standard output
from listing~\ref{lst:myname:outhello}, which was produced by the
program given in listing~\ref{lst:myname:helloword}.

\begin{lstlisting}[language={},caption={Standard output},label={lst:myname:outhello}]
Hello world everybody
\end{lstlisting}

\begin{lstlisting}[language={Python},caption={My first program},label={lst:myname:helloword}]
#! /usr/bin/env python

import sys

def hello(name):
    print 'Hello world %s' % (name)

if __name__ == '__main__':
    hello(sys.argv[1] if len(sys.argv) > 1 else 'everybody')
\end{lstlisting}

% Table inclusion

It is also possible to include tables, as done with
table~\ref{tbl:myname}. The numbering is done using another counter.

\begin{table}[ht]
  \centering
  \begin{tabular}{|c|}
  \hline
  Line 1 \\
  Line 2 \\
  ... \\
  Line n \\
  \hline
  \end{tabular}
  \caption{Table caption}
  \label{tbl:myname}
\end{table}

% Algorithm description

Finally, we include the \texttt{algorithm} package which allow to
include algorithm descriptions (see~\ref{alg:myname} for example).

\begin{algorithm}
\begin{algorithmic}[1]
  \Statex \textbf{Input:} $x$
  \Statex \textbf{Output:} $x+1$
  \State Add one!
\end{algorithmic}
\caption{My pretty algo}\label{alg:myname}
\end{algorithm}

\bibliography{MyName/biblio}

% Some typography nits
%
% You must not have any space between your text and the \footnote command.
%
% You must add an "espace insécable" ~ before references~\ref{...} and
% citation~\cite{...} commands.
